\documentclass[a4paper, 11pt]{article}

\voffset -0cm
\hoffset 0.0cm
\textheight 23cm
\textwidth 16cm
\topmargin 0.0cm
\oddsidemargin 0.0cm
\evensidemargin 0.0cm

\usepackage{epsfig}  
\usepackage{setspace}
\usepackage{fancyheadings}
\usepackage{amsmath}
\usepackage{amssymb}
\usepackage{graphicx}
\usepackage{url}
\usepackage[english, linesnumbered, ruled, vlined]{algorithm2e}

\title{}
\author{}
\date{}

\newtheorem{qu}{Question}

\begin{document}

\begin{center}
	\LARGE \textbf{Project: ``Volumetric analysis''}
\end{center}

\section*{Introduction} 

The objective of this project is to implement and evaluate some
volumetric analysis tools of digital objects. 

We expect from you:
\begin{itemize}
\item A short report with answers to the "formal" questions and a
  description of  your implementation choices and results.
\item A C++ project (\texttt{CMakeLists.txt} plus several
  \textbf{commented} cpp program files).
\end{itemize}


\section{Step 1: Distance Transformation}

Let us consider a 2D binary object $X$ (for instance, a parametric or implicit
shape from \textsc{DGtal} digitized at a given resolution $h$). 

\begin{qu}
  Using \texttt{DistanceTransformation} and \texttt{VoronoiMap}
  classes available in the library, implement tool that compute both
  distance transformation and Voronoi map of $X$.
\end{qu}

\section{Step 2: Discrete $\lambda$-medial Axis}

In computational geometry, the $\lambda$-medial axis is an
approximation of the medial axis of a continuous shape with both
geometric and topological guarantees \cite{Chazal2005}. Its definition
can be described as follows: given a point $x$ in the plane, let $\delta$
be the closest distance between $x$ and $\partial X$ (the boundary of
$X$). Let $S_x=\{s_i\}$ be the set of points in $\partial X$ such that
\begin{equation}
  d(x,s_i)=\delta
\end{equation}
If $S$ contains more than one point, $x$ belongs to the continuous
medial axis of $X$.  Now, a point $x$ belongs to the $\lambda-$medial
axis of $X$ if and only if the radius of the \emph{minimum enclosing
  circle} of $S_x$ is greater than $\lambda$.

In digital geometry, we will use the following approximation
\begin{enumerate}
\item At a point $p\in X$, we collect closest background points $S_p$
  from the VoronoiMap in a 3x3 window around $p$. Points in this set
  may not be exactly equidistant to $p$, this is where the approximation comes
  from.
\item We compute the minimum enclosing circle of $S_p$ and $p$ belongs
  to   the discrete $\lambda-$medial axis if the radius is greater
  than a given paramter $\lambda$
\end{enumerate}


\begin{qu}
  Implement a function that computes the minimum enclosing circle of a
  point set (see Appendix).
\end{qu}



\begin{qu}
  Implement a function that computes the $\lambda-$medial axis of
  $X$. Evaluate the quality of the medial representation according to
  paramter $\lambda$ and gridstep $h$.
\end{qu}

\section{Step 3: Thickness function}

Thickness function is an important tool for the analysis of porous
material. The function can be defined as follows for $p\in X$:
\begin{equation}
  \tau(p) = \max \{ r~|~ \forall B(c,r)\subset X, p\in
  B(c,r)\}
\end{equation}
In other words, the thickness at $p$ is the radius of the largest
circle containing $p$ inside the shape $X$\footnote{$B(c,r)$ is the Euclidean ball with center $c$ and
    radius $r$.}. Instead of checking all
balls contained in $X$, it is sufficient to only consider balls from
the medial axis of $X$.

\begin{qu}
  Implement a function that computes the thickness function of a shape
  $X$. First start by a $\lambda-$medial axis approximation and  use
  the obtained medial balls to compute the thickness for all points
  $p\in X$.
\end{qu}


\begin{qu}
  Export the thickness function as a graph with the abscissa is the
  thickness value and the ordinate, the number of pixels with
  thickness less than $x$. How can this spectra characterize the
  geometry of the shape. Can you guess what this has been used on
  porous material analysis ?
\end{qu}





\appendix
\section{Minimum Enclosing Circle}

The following section describes a randomized algorithm for the minimum
enclosing circle problem whose expected time is linear
\cite{Welzl1991}.  Given a set of points $P$, the following recursive
algorithm computes the minimum enclosing circle $B$ containing all
points $P$ and the set $R$ of points on $B$. The code is given for
points in $\mathbb{R}^d$ and must be calls with $R=\emptyset$.


\begin{algorithm}[Hhtbp]
  \KwIn{$P\subset\mathbb{R}^d$, $R\subset\mathbb{R}^d$}
  \KwResult{the minimum enclosing circle $B$}
  %
  \If{$P=\emptyset$ or $|R|=d+1$}
      {compute $B$ which is uniquely
        defined by  points in $R$\;
        \Return $B$\;
      }
      \Else{
        Choose $p\in P$ at random\;
        $B$ = minimumEnclosingCircle( $P\setminus \{p\}$, $R$)\;
        \If{$p\not\in B$}
           {
             $B$ = minimumEnclosingCircle( $P\setminus\{p\}$,$R\cup \{p\}$)\;
           }
           }
  \Return $B$
  
  \caption{minimumEnclosingCircle($P$,$R$)}
  \label{algo:main}
\end{algorithm}




\bibliographystyle{alpha}
\bibliography{refs}


\end{document}
