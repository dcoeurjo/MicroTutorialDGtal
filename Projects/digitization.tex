\documentclass[a4paper, 11pt]{article}

\voffset -0cm
\hoffset 0.0cm
\textheight 23cm
\textwidth 16cm
\topmargin 0.0cm
\oddsidemargin 0.0cm
\evensidemargin 0.0cm

\usepackage{epsfig}  
\usepackage{setspace}
\usepackage{fancyheadings}
\usepackage{amsmath}
\usepackage{amssymb}
\usepackage{graphicx}
\usepackage{url}

\title{}
\author{}
\date{}

\newtheorem{qu}{Question}

\begin{document}

\begin{center}
	\LARGE \textbf{Project: ``Mesh digitization with topological consistency''}
\end{center}

\section*{Introduction} 

The objective of this project is to implement and evaluate a mesh
digitization process in \textsc{DGtal}.

We expect from you:
\begin{itemize}
\item A short report with answers to the "formal" questions and a
  description of the your implementation choices and results.
\item A C++ project (\texttt{CMakeLists.txt} plus several
  \textbf{commented} cpp program files).
\end{itemize}


\section{Manifold case}

The main problem is to convert a triangular mesh $\mathcal{M}$ into a
digital object while maintaining some topological properties.

The input is thus a triangular mesh (in OFF format for instance, see
\texttt{Mesh} and \texttt{MeshReader} classes in \textsc{DGtal}) and a
grid step $h$ and the output should be a 3D digital object.


\begin{qu}
  Let us suppose that $\mathcal{M}$ (and its OFF file) is a correct
  combinatorial manifold (no holes). Implement a digitization process
  of the surface parametrized by $h$.
\end{qu}


At this point, several options exist (different digitization models,
different parameters...) with several outcomes (create $k-$separating
digital surfaces for some $k\in\{6,18,26\}$...). We have added some
references at the end of the document but what ever your
implementation choices, we expect you to clearly justify these choices
in the report.

To be more precise, when implementing the digitization process, we
would evaluate the algorithm in terms of:
\begin{itemize}
\item Topologically consistency (hole free if the input is a manifold)
\item Scalablility (performances when either the resolution or the
  number of triangles increases)
\item Invariance to (integer) translations, symmetry w.r.t. to grid
  axis or flip of triangle normal vectors
\end{itemize}

\section{From digital surface to digital objects}

\begin{qu}
  Implement an algorithm that will fill the digital surface interior
  to obtain a digital object. 
\end{qu}

The algorithm should
\begin{itemize}
\item handle multiply connected objects (or nested objects).
\item be consistent with the digital surface topology 
\end{itemize}


\section{Non-manifold case}

We now consider that the input triangulation is given by a collection
of triangles without any topological relationship between them. Such
input is not a manifold anymore and can even contain intersection
triangles.

We suppose that gaps between triangles come from a erroneous
reconstruction step of a manifold. Hence, we can assume that there is
a (unknown) parameter $\epsilon$ such that all duplicate vertices lie
in a ball of radius $\epsilon$.

Many alternatives exist to digitize such surface while correcting the
topology (mathematical morphology, distance based, topological
thinning...).

\begin{qu}
  Design and implement an algorithm to digitize such set of
  triangles. Please clearly discuss your choices (outcomes/drawbacks,
  topological guarantees).
\end{qu}

\nocite{Cohen-Or1997,Schwarz2010,Laine2013}

\bibliographystyle{alpha}
\bibliography{refs}

\end{document}
