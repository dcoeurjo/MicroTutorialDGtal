\documentclass[a4paper, 11pt]{article}

\voffset -0cm
\hoffset 0.0cm
\textheight 23cm
\textwidth 16cm
\topmargin 0.0cm
\oddsidemargin 0.0cm
\evensidemargin 0.0cm

\usepackage{epsfig}  
\usepackage{setspace}
\usepackage{fancyheadings}
\usepackage{amsmath}
\usepackage{amssymb}
\usepackage{graphicx}
\usepackage{url}

\title{}
\author{}
\date{}

\newtheorem{qu}{Question}

\begin{document}

\begin{center}
	\LARGE \textbf{Project: ``Length estimation''}
\end{center}

\section*{Introduction}

The objective of this project is to implement and evaluate a length estimator
based on an integration process.


We expect from you:
\begin{itemize}
\item A short report with answers to the ``formal'' questions and a
  description of your implementation choices and results.
\item A C++ project (\texttt{CMakeLists.txt} plus several
  \textbf{commented} cpp program files).
\end{itemize}


\section{Length estimator}

Let $X \subset \mathbb{R}^2$ be a convex shape and $\partial X$ be its boundary. 
In order to estimate the perimeter of $X$, we will focus on a discrete version of 
\begin{equation}
\label{eq:length}
\mathcal{L}(X) = \int_{\partial X} \vec{t}(s) \cdot ds.
\end{equation}

Let us consider the Gauss digitization $Z \subset \mathbb{Z}^2$ of $X$ at gridstep $h$:
\begin{displaymath}
  Z = Dig(X,h) = X\cap (h\cdot \mathbb{Z}^2).
\end{displaymath}

The contour of $Z$, which is merely denoted by $\partial Z$, is a sequence of 1-cells.  
%Let us consider the whole set of maximal 
%digital straight segments. 
%We define the unit tangent vector \hat{t} associated 
%to a 0-cell $z$ as the normalized direction vector of the maximal digital straight segment that is the 
%most centered on $z$. 
We define the unit tangent vector $\hat{\vec{t}}$ associated to a 1-cell as the mean
vector of the normalized direction vector of the most centered maximal digital straight segments
computed at its two end points. We also define an elementary tangent vector $\vec{t}_{e}$
associated to a 1-cell as the unit vector colinear with this 1-cell. 

Hence, the discrete version of (\ref{eq:length}) is:
\begin{equation}
\label{eq:length-estimator}
\hat{\mathcal{L}(X)} = h \cdot \Big( \sum_{c \in \partial Dig(X,h)}  \hat{\vec{t}}(c) . \vec{t}_{e}(c) \Big). 
\end{equation}

\begin{qu}
 Let us consider a period of a digital straight line of slope $a/b$. 
Give the estimation of the length of such period according to (\ref{eq:length-estimator}). 
\end{qu}

\begin{qu}
Implement the estimator (\ref{eq:length-estimator}). You can use \textsc{DGtal} and 
more precisely the classes {\tt GridCurve}, {\tt ArithmeticalDSSComputer} and 
{\tt SaturatedSegmentation} in order to compute the whole set of maximal digital straight 
segments. You may found several test or example files where such a computation is done. 
\end{qu}

\section{Multigrid Analysis}


\begin{qu}
  Perform a multigrid analysis of $\hat{\mathcal{L}(X)}$. 
  \begin{itemize}
  \item Implement a function that constructs the digitization of a disk 
$D: \sqrt{x^2 + y^2} = 1$ at a gridstep $h$.  
  \item Output $| \hat{\mathcal{L}(X)} - 2\pi |$ values for $h$ tending to 0.
  \item Plot error graphs $h \times Error$ in logscale using e.g. \texttt{gnuplot}.
  \end{itemize}	 
\end{qu}	

\begin{qu} 
We guess that $\hat{\mathcal{L}(X)}$ has an error in $O(h^{\alpha})$, $\alpha \in \mathbb{R}$. 
\begin{itemize}
\item Recall the theoritical bounds we can prove on $\alpha$. 
\item Experimentally, can you estimate such $\alpha$ for disks? (Hint: the slope $\beta$ of 
linear fitting in logscale gives you the exponent of something on $x^\beta$). 
\end{itemize}
\end{qu}	

\section{Extra works}

\begin{qu}
Do you get similar values for $\alpha$ when working with other shapes than disks, like ellipses, squares, and so on ?
\end{qu}

\begin{qu}
  Perform a complete multigrid convergence evaluation with comparison
  to both expected quantities (available in \textsc{DGtal} for
  implicit shapes, see the documentation) and estimated ones from other
  estimators (e.g. estimators based on a greedy DSS segmentation, see the 
 documentation).
\end{qu}


\end{document}
