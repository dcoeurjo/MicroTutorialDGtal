\documentclass[a4paper, 11pt]{article}

\voffset -0cm
\hoffset 0.0cm
\textheight 23cm
\textwidth 16cm
\topmargin 0.0cm
\oddsidemargin 0.0cm
\evensidemargin 0.0cm

\usepackage{epsfig}  
\usepackage{setspace}
\usepackage{fancyheadings}
\usepackage{amsmath}
\usepackage{amssymb}
\usepackage{graphicx}
\usepackage{url}

%algo
\usepackage[english, linesnumbered, ruled, vlined]{algorithm2e}

\title{}
\author{}
\date{}

\newtheorem{qu}{Question}

\begin{document}

\begin{center}
	\LARGE \textbf{Project: ``Recognition of digital circle''}
\end{center}

\section*{Introduction}

The objective of this project is to implement a linear programming based algorithm
for the recognition of digital circles. 

We expect from you:
\begin{itemize}
\item A short report with answers to the ``formal'' questions and a
  description of your implementation choices and results.
\item A C++ project (\texttt{CMakeLists.txt} plus several
  \textbf{commented} cpp program files).
\end{itemize}


\section{Digital disk}

Let $I \subset \mathbb{Z}^2$ be a rectangular digital image. Let $Z \subset I$ be 
a digital set and $\bar{Z} = I \setminus Z$ be its complement set.

A digital set $Z$ is a \emph{digital disk} if and only if there exists a circle of 
center $\omega \in \mathbb{R}^2$ and of radius $\rho \in \mathbb{R}$ such that:  
\begin{equation}
  \left\{
  \begin{array}{l}
    \forall z \in Z, \: \| z - \omega \|^2 \leq \rho^2 \\
    \forall z \in \bar{Z}, \: \|  z - \omega \|^2 \geq \rho^2 \\
  \end{array}
  \right.
\end{equation}

%question lien disc de Gauss
\begin{qu}
Let us consider the Gauss digitization of a convex shape $X \subset \mathbb{R}^2$ at gridstep $h$:
\begin{displaymath}
  Dig(X,h) = X\cap (h\cdot \mathbb{Z}^2).
\end{displaymath}
Show that if $X$ is an Euclidean disk, then $Dig(X,h)$ is a digital disk. Show however that 
the converse is not true. 
\end{qu}

%question separation et determinant
\begin{qu}
There exists a unique circle passing by three digital points. Show that we can test whether another digital
point lies INSIDE, ON or OUTSIDE such a circle with integer-only computations and without explicitly computing 
its center and radius. You may have a look at ``in-circle test'', broadly used in computational geometry, e.g. 
to compute the Delaunay triangulation of a point set.  
\end{qu}

\begin{qu}
Implement a function that checks whether two given digital sets are separated by a given circle passing by three
digital points or not.  
\end{qu}

\section{Linear programming}

Let us now consider algorithm \ref{algo:main} (which uses routine \ref{algo:rec}). 
It is a randomized and recursive algorithm that checks whether two point sets
 are separable by a circle in expected linear-time (see \cite{Seidel1991} or \cite{Berg2000}[chapter4]). 

The union of the inner point set, denoted by $S^-$, and the outer point set, denoted by $S^+$, is 
merely denoted by $S$. All the points of $S$ are numbered from $1$ to $|S|$, the size of $S$.    
The idea consists in maintaining a separating circle while iterating over the points $s_i \in S$ 
for $i$ from $1$ to $|S|$. 
For each point $s_i$, three cases may occur:  
\begin{itemize}
 \item if it belongs to $S^-$ (resp. $S^+$) and it is located INSIDE (resp. OUTSIDE) or ON 
the current separating circle, there is nothing to do. 
 \item Otherwise (lines 6-9 of algorithm \ref{algo:rec}): 
 \begin{enumerate}
   \item Either the two input sets are not circularly separable at all,  
   \item or there exists a separating circle passing by $s_i$. 
 \end{enumerate}
\end{itemize}
In the aim of deciding between these last two alternatives, the set of possible separating 
circles is restricted to circles passing by $s_i$ and the same algorithm is recursively called
from $1$ to $i$ (line 9 of algorithm \ref{algo:rec}). 
At each recursive call, the set of possible separating circles is restricted so that the base case 
involves a unique circle passing by three given points and consists in checking whether it separates 
$S^-$ from $S^+$ or not (lines 11-17 of algorithm \ref{algo:rec}).  


%% \begin{algorithm}[Hhtbp]
%%   \KwIn{$S^-, S^+ \subset \mathbb{Z}^2$, the inner and outer point sets}
%%   \KwResult{``true'' if $S$ and $T$ are circularly separable, ``false'' otherwise}
%%   \KwOut{$a, b, c \in \mathbb{Z}^2$, three digital points caracterizing a separating circle if ``true''}
%%   %
%%   \tcp{initialisation}
%%   Construct the set of $S = S^- \cup S^+$ \; 
%%   \tcp{points of $S$ are numbered from $0$ to $|S|-1$, $|S|$ is the size of the set}
%%   Randomly permute the points of $S$ \;
%%   %cas particulier
%%   Initialize $a, b, c$ with three points of $S$ \tcp*{we assume that $|S| > 3$} 
%%   %
%%   \tcp{main loop}
%%   $i \leftarrow 3$ \; 
%%   areSeparable $\leftarrow$ TRUE \; 
%%   \While { \emph{areSeparable} and $i < |S|$ }{
%%     %
%%     \If{($s_i \in S^-$ and $s_i$ is outside the circle passing by $a,b,c$) \\ 
%%       or ($s_i \in S^+$ and $s_i$ is inside the circle passing by $a,b,c$) }{
%%       %
%%       $a \leftarrow s_i$ \; 
%%       areSeparable $\leftarrow$ areCircularlySeparableWithPoint($S^-$, $S^+$, $S$, $a$, $b$, $c$) \; 
%%       %
%%     }
%%     \Return areSeparable \; 
%%   }
%%   %
%%   \caption{areCircularlySeparable($S^-$, $S^+$, $a$, $b$, $c$)}
%%   \label{algo:3}
%% \end{algorithm}

%% \begin{algorithm}[Hhtbp]
%%   \KwIn{$S^-, S^+ \subset \mathbb{Z}^2$, the inner and outer point sets}
%%   \KwResult{``true'' if $S$ and $T$ are circularly separable, ``false'' otherwise}
%%   \KwOut{$a, b, c \in \mathbb{Z}^2$, three digital points caracterizing a separating circle if ``true''}
%%   %
%%   Initialize $b$ and $c$ with two points of $S$ \;  
%%   %
%%   \tcp{main loop}
%%   $i \leftarrow 3$ \; 
%%   areSeparable $\leftarrow$ TRUE \; 
%%   \While { \emph{areSeparable} and $i < |S|$ }{
%%     %
%%     \If{($s_i \in S^-$ and $s_i$ is outside the circle passing by $a,b,c$) \\ 
%%       or ($s_i \in S^+$ and $s_i$ is inside the circle passing by $a,b,c$) }{
%%       %
%%       $a \leftarrow s_i$ \; 
%%       areSeparable $\leftarrow$ areCircularlySeparableWithPoint($S^-$, $S^+$, $S$, $a$, $b$, $c$) \; 
%%       %
%%     }
%%     \Return areSeparable \; 
%%   }
%%   %
%%   \caption{areCircularlySeparableWithPoint($S^-$, $S^+$, $S$, $a$, $b$, $c$)}
%%   \label{algo:2}
%% \end{algorithm}

\begin{algorithm}[Hhtbp]
  \KwIn{$S^-, S^+ \subset \mathbb{Z}^2$, the inner and outer point sets \\
  $p_1, p_2, p_3 \in \mathbb{Z}^2$, three points characterizing a circle }
  \KwResult{``true'' if $S^-$ and $S^+$ are circularly separable, ``false'' otherwise}
  \KwOut{$p_1, p_2, p_3$, three points caracterizing a separating circle if ``true''}
  %
  \tcp{initialisation step}
  Construct the set of $S = S^- \cup S^+$ \; 
  Randomly permute the points of $S$ \;
  \tcp{points of $S$ are numbered from $1$ to $|S|$, $|S|$ is the size of the set}
  Initialize $p_1, p_2, p_3$ with three points of $S$ \tcp*{we assume here that $|S| > 3$}
  \tcp{recursive step}
  \Return areCircularlySeparable($S^-$, $S^+$, $S$, $|S|$, $p_1$, $p_2$, $p_3$, $3$) \; 
  %
  \caption{areCircularlySeparable($S^-$, $S^+$, $p_1$, $p_2$, $p_3$)}
  \label{algo:main}
\end{algorithm}

\begin{algorithm}[Hhtb]
  \KwIn{$S^-, S^+ \subset \mathbb{Z}^2$, the inner and outer point sets, $S = S^- \cup S^+$ \\
  $n$, number of points of $S$ to process ($1 \leq n \leq |S|$) \\
  $p_1, p_2, p_3 \in \mathbb{Z}^2$, three points characterizing a circle \\
  $k$, number of variable points among $\{p_1,p_2,p_3\}$ ($0 \leq k \leq 3$)}
  \KwResult{``true'' if $S^-$ and $S^+$ are circularly separable, ``false'' otherwise}
  \KwOut{$p_1, p_2, p_3$, three points caracterizing a separating circle if ``true''}
  %
  areSeparable $\leftarrow$ TRUE \; 
  \If{$k > 0$}{
    for $l$ from $1$ to $k$, initialize $p_l$ with a point of $S$ \;  
    %
    $i \leftarrow 1$ \; 
    \While { \emph{areSeparable} and $i < n$ }{
      %
      \If{($s_i \in S^-$ and $s_i$ is strictly OUTSIDE the circle passing by $p_1, p_2, p_3$) \\ 
        or ($s_i \in S^+$ and $s_i$ is strictly INSIDE the circle passing by $p_1, p_2, p_3$) }{
        %
        $p_k \leftarrow s_i$ \; 
        areSeparable $\leftarrow$ areCircularlySeparable($S^-$, $S^+$, $i$, $p_1$, $p_2$, $p_3$, $k-1$) \; 
        %
      }
      $i \leftarrow i + 1$ \; 
    }    
  }
  %
  \Else{
    $i \leftarrow 1$ \; 
    \While { \emph{areSeparable} and $i < n$ }{
      %
      \If{($s_i \in S^-$ and $s_i$ is strictly OUTSIDE the circle passing by $p_1, p_2, p_3$) \\ 
        or ($s_i \in S^+$ and $s_i$ is strictly INSIDE the circle passing by $p_1, p_2, p_3$) }{
        %
        areSeparable $\leftarrow$ FALSE \; 
        %
      }
      $i \leftarrow i + 1$ \; 
    }    
  }
  \Return areSeparable \; 
  %
  \caption{areCircularlySeparable($S^-$, $S^+$, $S$, $n$, $p_1$, $p_2$, $p_3$, $k$)}
  \label{algo:rec}
\end{algorithm}


%% \begin{qu}
%% When its last parameter is $0$, what does algorithm \ref{algo:rec} ? Compare to the previous question.
%% When its last parameter is $k \in \{1,2,3\}$, what does algorithm \ref{algo:rec} ? 
%% \end{qu}

%question implementation avec et sans randomization
\begin{qu}
Implement algorithms \ref{algo:main} and \ref{algo:rec}. Provide test files. 
\end{qu}

\section{Experiments}

%% Let us consider now the contour $C$ of a digital set $Z$. If $Z$ is connected
%% and does not have any hole, its contour consists of a circular sequence of 1-cells. 
%% Each 1-cell of the contour is the common edge of a 2-cell whose center is a point of $Z$ 
%% and a 2-cell whose center is a point of $\bar{Z}$. Let us denote by $C^-$ (resp. $C^+$)
%% the set of digital points of $Z$ (resp. $\bar{Z}$) that are the center of a 2-cell 
%% incident to a 1-cell of $C$.  
  
%% A contour $C$ is a \emph{digital circle} if and only if there exists a circle of 
%% center $\omega \in \mathbb{R}^2$ and of radius $\rho \in \mathbb{R}$ such that:  
%% \begin{equation}
%%   \left\{
%%   \begin{array}{l}
%%     \forall z \in C^-, \: \| z - \omega \|^2 \leq \rho^2 \\
%%     \forall z \in C^+, \: \| z - \omega \|^2 \geq \rho^2 \\
%%   \end{array}
%%   \right.
%% \end{equation}

%% \begin{qu}
%% Implement a function that checks whether $C$ is a digital circle or not. 
%% \end{qu}


\begin{qu}
In order to check whether two connected digital sets $Z$ and $\bar{Z}$ are circularly separable, 
it is enough to consider only the digital boundaries of $Z$ and $\bar{Z}$. Implement a function
that takes as input the common contour of $Z$ and $\bar{Z}$ and that checks whether $Z$ and $\bar{Z}$
are circularly separable or not. You may use \textsc{DGtal} and more precisely the class {\tt GridCurve}
that can return the set of boundary digital points (see e.g. {\tt IncidentPointsRange}).    
\end{qu}

\begin{qu}
  Perform a running time analysis of your recognition function.  
  \begin{itemize}
  \item Implement a function that constructs the contour of a disk of a given radius.  
  \item Output the running time of your recognition function for disks of increasing radius. 
  \item Plot the graph of the running times against the size of the contour.
  \item Do you observe the expected linear-time complexity ?
  \end{itemize}	 
\end{qu}	

\section{Extra works}

\begin{qu}
Modify your recognition procedure in order to have an on-line algorithm, 
which takes input points two by two (one belonging to the boundary of $Z$ and one 
belonging to the boundary of $\bar{Z}$)
 and updates the current separating circle on the fly.
What is the time complexity of your procedure ?
\end{qu}

\begin{qu}
Use your on-line procedure to partition a contour into digital circular arcs and
 to compute the whole set of maximal digital circular arcs.  
\end{qu}

\nocite{Seidel1991,Berg2000}
\bibliographystyle{alpha}
\bibliography{refs}


\end{document}
