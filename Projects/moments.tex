\documentclass[a4paper, 11pt]{article}

\voffset -0cm
\hoffset 0.0cm
\textheight 23cm
\textwidth 16cm
\topmargin 0.0cm
\oddsidemargin 0.0cm
\evensidemargin 0.0cm

\usepackage{epsfig}  
\usepackage{setspace}
\usepackage{fancyheadings}
\usepackage{amsmath}
\usepackage{amssymb}
\usepackage{graphicx}
\usepackage{url}

\title{}
\author{}
\date{}

\newtheorem{qu}{Question}

\begin{document}

\begin{center}
	\LARGE \textbf{Project: ``Multigrid Convergence of Geometrical moments''}
\end{center}

\section*{Introduction}

The objective of this project is to implement and evaluate geometrical
moments computation on digital data.


We expect from you:
\begin{itemize}
\item A short report with answers to the "formal" questions and a
  description of our implementation choices and results.
\item A C++ project (\texttt{CMakeLists.txt} plus several
  \textbf{commented} cpp program files).
\end{itemize}


\section{Properties of moments}
We are interest in the analysis of geometrical moment of order
$m_{p,q}$ defined as follows for $X\subset\mathbb{R}^2$:
\begin{displaymath}
 m_{p,q}(X) = \int\int_X x^py^q\,dxdy
\end{displaymath}

On a digital object $Z\in\mathbb{Z}^2$, we will use the following approximation:
\begin{displaymath}
\hat{m}_{p,q}(Z) = \sum_{(i,j)\in Z} i^pj^q
\end{displaymath} 

Similarly, we will also consider {\bf central geometrical moments} defined by:

\begin{displaymath}
 \mu_{p,q}(X) = \int\int_X (x-\mu_x)^p(y-\mu_y)^q\,dxdy
\end{displaymath}
\begin{displaymath}
 \hat{\mu}_{p,q}(Z) =  \sum_{(i,j)\in Z} (i-\mu_i)^p(p-\mu_j)^q
\end{displaymath}
where $(\mu_x,\mu_y)$ (resp. $(\mu_i,\mu_j)$) is the centroid of X (resp. Z).

\begin{qu}
 Express $(\mu_x,\mu_y)$ coordinates as function of $m_{a,b}$ for some $a,b\in\mathbb{Z}$
\end{qu}

\begin{qu}
 Express first $\mu_{p,q}$ central moments ($p+q\leq 2$) as functions of $m_{p,q}$.
\end{qu}


\begin{qu}
  Geometrical moments are not scale-invariant. We consider a
  scaling of $X$ by a factor $k$ (denoted $k\cdot X$), express
  $m_{p,q}(k\cdot X)$ as a function of $m_{p,q}(X)$
\end{qu}

\noindent Similarly, you also have $\mu_{p,q}(k\cdot X)$ as a function
of $\mu_{p,q}(X)$.  We can use the previous result to design moments
$\eta_{p,q}$ with scale invariant properties.

\begin{qu}
\begin{itemize}
\item Express $m_{0,0}(k\cdot X)$ from $m_{0,0}(X)$.
\item Find the $\alpha$ such that 
\begin{displaymath}
\frac{m_{p,q}(k\cdot X)}{m_{0,0}(k\cdot X)^\alpha} = \frac{m_{p,q}(X)}{m_{0,0}(X)^\alpha}
\end{displaymath}
\item Define $\eta_{p,q}$ as a function of $\mu_{p,q}$ and $\mu_{0,0}$. Conclude on the fact that $\eta_{p,q}$ are translation and scale invariant.
\end{itemize}
\end{qu}


\section{Multigrid Analysis}

Here, we evaluate the multigrid convergence of $\hat{m}_{pq}$
estimators. First, remember that we consider here Gauss digitization
at gridstep $h$ of $X\subset \mathbb{R}^2$:
\begin{displaymath}
  Z = Dig(X,h) = \left(\frac{1}{h}\cdot X\right)\cap \mathbb{Z}^2 = X\cap (h\cdot \mathbb{Z}^2)
\end{displaymath}




As discussed above, we denote
\begin{displaymath}
\hat{m}_{pq}(Z) = \sum_{(i,j)\in Z} i^pj^q
\end{displaymath}
and define 
\begin{displaymath}
\hat{m}_{pq}(Z,h) = \hat{m}_{pq}(h\cdot Z)
\end{displaymath}




\begin{qu}
From the definition and using the result of Question 4, express
$\hat{m}_{pq}(Z,h)$ as a function of $\hat{m}_{pq}(Z)$ and $h$.
\end{qu}

\begin{qu}
Implement in \textsc{DGtal} geometrical moments  $\hat{m}_{pq}(Z,h)$
computations in 2D. In this multigrid context, please consider the
digitization of a parametric shape (\emph{e.g.} an ellipse).
\end{qu}



\begin{qu}
	Perform a multigrid analysis of $\hat{m}_{pq}(Dig(X,h), h)$
	 \begin{itemize}
	 \item Implement function which constructs the digitization of
           an Euclidean ellipse $E: \frac{x^2}{a^2} + \frac{y^2}{b^2}
           = 1$ at grid step $h$ 
	 \item For the first moments ($p+q\leq 3$), output $|
           \hat{m}_{pq}(Dig(X,h), h) - m_{pq}(E)|$ values for $h$
           tending to 0.
	 \item Plot error graphs $h\times Error$ in logscale using \texttt{gnuplot}
\end{itemize}	 
\end{qu}	

See Appendix A for first geometrical moments of the ellipse.


\begin{qu} 
We guess that $\hat{m}_{pq}$ has an error in $O(h^{\alpha_{pq}})$ for
some $\alpha_{pq}\in\mathbb{R}$ depending only on $p$ and
$q$. Experimentally, can you estimate such $\alpha_{pq}$ for first
moments ? (Hint: the slope $\beta$ of linear fitting in logscale gives
you the exponent of something on $x^\beta$). Can you guess the general
form for the error of $m_{pq}$.
\end{qu}	

So far, we saw in the lectures that $\hat{m}_{00}$ is convergent with
speed at least $O(h)$ ([Gauss] with general convex hypothesis on
$X$). Hence, we have $\alpha_{00} = 1$. Adding hypothesis on $\partial
X$ (e.g. being $C^3$), we have $\alpha_{00} = \frac{15}{11}-\epsilon$
[Huxley]


\section{Extensions}

\begin{qu}
  In dimension 3, are the convergence speeds different ?
\end{qu}


\begin{qu}
  Are moments $m_{pq}$ or $\hat{m}_{pq}$ rotational invariant ? Can
  you construct rotational invariant shape descriptor from $m_{pq}$ or
  $\hat{m}_{pq}$ ?
\end{qu}



\appendix
\section{Geometrical moments of a General Ellipse}

Notations:
\begin{itemize}
\item $a$ length of the semi-major axis
\item $b$ length of the semi-minor axis
\item $x_0$, $y_0$ coordinate of the center of the ellipse
\item $\lambda$ angle of the major axis with the $x-$axis
\end{itemize}


\begin{align}
m_{00} &= \pi ab\\
m_{10} &= \pi ab x_0\\
m_{01} &= \pi ab y_0\\
m_{20} &= \pi ab\left( \frac{a^2\cos^2\lambda + b^2\sin^2\lambda}{4} + x_0^2\right)\\
m_{02} &= \pi ab\left( \frac{a^2\sin^2\lambda + b^2\cos^2\lambda}{4} + y_0^2\right)\\
m_{11} &= \pi ab\left( \frac{(a^2-b^2)\cos\lambda\sin\lambda}{4} + x_0y_0\right)\\
m_{30} &= \pi ab\left( \frac{3x_0(a^2\cos^2\lambda + b^2\sin^2\lambda)}{4} + x_0^3\right)\\
m_{03} &= \pi ab\left( \frac{3y_0(a^2\sin^2\lambda + b^2\cos^2\lambda)}{4} + y_0^3\right)\\
m_{21} &= \pi ab\left( \frac{y_0(a^2\cos^2\lambda + b^2\sin^2\lambda)}{4} +  \frac{x_0(a^2-b^2)\sin\lambda\cos{\lambda}}{2} + x_0^2y_0\right)\\
m_{12} &= \pi ab\left( \frac{x_0(a^2\sin^2\lambda + b^2\cos^2\lambda)}{4} +  \frac{y_0(a^2-b^2)\sin\lambda\cos\lambda}{2} + x_0y_0^2\right)
\end{align}

\end{document}
