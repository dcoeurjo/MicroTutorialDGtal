\documentclass[a4paper, 11pt]{article}

\voffset -0cm
\hoffset 0.0cm
\textheight 23cm
\textwidth 16cm
\topmargin 0.0cm
\oddsidemargin 0.0cm
\evensidemargin 0.0cm

\usepackage{epsfig}  
\usepackage{setspace}
\usepackage{fancyheadings}
\usepackage{amsmath}
\usepackage{amssymb}
\usepackage{graphicx}
\usepackage{url}

%algo
\usepackage[english, linesnumbered, ruled, vlined]{algorithm2e}

\title{}
\author{}
\date{}

\newtheorem{qu}{Question}

\begin{document}

\begin{center}
	\LARGE \textbf{Project: ``Recognition of digital plane''}
\end{center}

\section*{Introduction}

The objective of this project is to implement an linear programming based algorithm
for the recognition of digital planes. 

We expect from you:
\begin{itemize}
\item A short report with answers to the ``formal'' questions and a
  description of your implementation choices and results.
\item A C++ project (\texttt{CMakeLists.txt} plus several
  \textbf{commented} cpp program files).
\end{itemize}


\section{Digital plane}

In the sequel, let us consider the following definition: 
a digital set $Z$ is a \emph{digital plane} if and only if 
there exists a normal vector $N(a,b,c) \in \mathbb{Z}^3$ and 
a bound $\mu \in \mathbb{Z}$ such that:
\begin{equation}
\label{eq:arith-def}
    \forall z \in Z, \: \mu \leq N \cdot z \leq \mu + \max{(|a|,|b|,|c|)}
\end{equation}
(where $\cdot$ denote the scalar product). 

\begin{qu}
We assume now that $0 \leq a \leq b < c$. 
Show that \ref{eq:arith-def}) is equivalent to: 
\begin{equation}
\forall z \in Z, \:
  \left\{
  \begin{array}{l}
     N \cdot z \leq \mu  \\
     N \cdot (z+(0,0,1)) \geq \mu  \\
  \end{array}
  \right.
\end{equation}
\end{qu}


%question separation et determinant
\begin{qu}
There exists a unique (Euclidean) plane passing through three digital points. Show that we can test whether another digital
point lies BELOW, ON or ABOVE such a plan with integer-only computations and without explicitly computing 
its center and radius. You may have a look at ``orientation test'' or ``which-side test'', broadly used in computational geometry. 
\end{qu}

\begin{qu}
Implement a function that checks whether two given digital sets are separated by a given plane passing through three
digital points or not.  
\end{qu}

Let us now consider Algorithm \ref{algo:main} (which uses Algorithm \ref{algo:rec}). 
It is a randomized and recursive algorithm that checks whether two point sets
 are separable by a plane in expected linear-time. 
%refs

The union of the bottom point set, denoted by $S^-$, and the top point set, denoted by $S^+$, is 
merely denoted by $S$. All the points of $S$ are numbered from $1$ to $|S|$, the size of $S$.    
The idea consists in maintaining a separating plane while iterating over the points $s_i \in S$ from $1$ to $|S|$. 
For each point $s_i$, three cases may occur:  
\begin{itemize}
 \item if it belongs to $S^-$ (resp. $S^+$) and it is located BELOW (resp. ABOVE) or ON 
the current separating plane, there is nothing to do. 
 \item Otherwise (lines 6-9 of algorithm \ref{algo:rec}): 
 \begin{enumerate}
   \item Either the two input sets are not separable by a plane at all,  
   \item or there exists a separating plane passing by $s_i$. 
 \end{enumerate}
\end{itemize}
In the aim of deciding between these last two alternatives, the set of possible separating 
planes is restricted to planes passing by $s_i$ and the same algorithm is recursively called
from $1$ to $i$ (line 9 of algorithm \ref{algo:rec}). 
At each recursive call, the set of possible separating planes is restricted so that the base case 
involves a unique plane passing by three given points and consists in checking whether it separates 
$S^-$ from $S^+$ or not (lines 11-17 of algorithm \ref{algo:rec}).  


\begin{algorithm}[Hhtbp]
  \KwIn{$Z \subset \mathbb{Z}^3$, the digital set \\
  $p_1, p_2, p_3 \in \mathbb{Z}^2$, three points characterizing a plane }
  \KwResult{``true'' if $Z$ is a digital plane, ``false'' otherwise}
  \KwOut{$p_1, p_2, p_3$, three points characterizing a separating plane if ``true''}
  %
  \tcp{initialisation step}
  Construct the set $S^- = Z$ and the set $S^+$ a copy of $Z$ translated by $(0,0,1)$ \; 
  Construct the set of $S = S^- \cup S^+$ \; 
  Randomly permute the points of $S$ \;
  \tcp{points of $S$ are numbered from $1$ to $|S|$, $|S|$ is the size of the set}
  Initialize $p_1, p_2, p_3$ with three points of $S$ \tcp*{we assume here that $|S| > 3$}
  \tcp{recursive step}
  \Return areSeparable($S^-$, $S^+$, $S$, $|S|$, $p_1$, $p_2$, $p_3$, $k$) \; 
  %
  \caption{areSeparable($Z$, $p_1$, $p_2$, $p_3$)}
  \label{algo:main}
\end{algorithm}

\begin{algorithm}[Hhtb]
  \KwIn{$S^-, S^+ \subset \mathbb{Z}^2$, the bottom and top point sets, $S = S^- \cup S^+$ \\
  $n$, number of points of $S$ to process ($1 \leq n \leq |S|$) \\
  $p_1, p_2, p_3 \in \mathbb{Z}^2$, three points characterizing a plane \\
  $k$, number of variable points among $\{p_1,p_2,p_3\}$ ($0 \leq k \leq 3$)}
  \KwResult{``true'' if $S^-$ and $S^+$ are separable by a plane, ``false'' otherwise}
  \KwOut{$p_1, p_2, p_3$, three points characterizing a separating plane if ``true''}
  %
  areSeparable $\leftarrow$ TRUE \; 
  \If{$k > 0$}{
    for $l$ from $1$ to $k$, initialize $p_l$ with a point of $S$ \;  
    %
    $i \leftarrow 1$ \; 
    \While { \emph{areSeparable} and $i < n$ }{
      %
      \If{($s_i \in S^-$ and $s_i$ is strictly ABOVE the plane passing by $p_1, p_2, p_3$) \\ 
        or ($s_i \in S^+$ and $s_i$ is strictly BELOW the plane passing by $p_1, p_2, p_3$) }{
        %
        $p_k \leftarrow s_i$ \; 
        areSeparable $\leftarrow$ areSeparable($S^-$, $S^+$, $i$, $p_1$, $p_2$, $p_3$, $k-1$) \; 
        %
      }
      $i \leftarrow i + 1$ \; 
    }    
  }
  %
  \Else{
    $i \leftarrow 1$ \; 
    \While { \emph{areSeparable} and $i < n$ }{
      %
      \If{($s_i \in S^-$ and $s_i$ is strictly ABOVE the plane passing by $p_1, p_2, p_3$) \\ 
        or ($s_i \in S^+$ and $s_i$ is strictly BELOW the plane passing by $p_1, p_2, p_3$) }{
        %
        areSeparable $\leftarrow$ FALSE \; 
        %
      }
      $i \leftarrow i + 1$ \; 
    }    
  }
  \Return areSeparable \; 
  %
  \caption{areSeparable($S^-$, $S^+$, $S$, $n$, $p_1$, $p_2$, $p_3$, $k$)}
  \label{algo:rec}
\end{algorithm}

%% \begin{qu}
%% When its last parameter is $0$, what does algorithm \ref{algo:rec} ? Compare to the previous question.
%% When its last parameter is $k \in \{1,2,3\}$, what does algorithm \ref{algo:rec} ? 
%% \end{qu}

%question implementation avec et sans randomization
\begin{qu}
Implement algorithms \ref{algo:main} and \ref{algo:rec}. Provide test files. 
\end{qu}

\section{Experiments}

%% Let us consider now the contour $C$ of a digital set $Z$. If $Z$ is connected
%% and does not have any hole, its contour consists of a circular sequence of 1-cells. 
%% Each 1-cell of the contour is the common edge of a 2-cell whose center is a point of $Z$ 
%% and a 2-cell whose center is a point of $\bar{Z}$. Let us denote by $C^-$ (resp. $C^+$)
%% the set of digital points of $Z$ (resp. $\bar{Z}$) that are the center of a 2-cell 
%% incident to a 1-cell of $C$.  
  
%% A contour $C$ is a \emph{digital plane} if and only if there exists a plane of 
%% center $\omega \in \mathbb{R}^2$ and of radius $\rho \in \mathbb{R}$ such that:  
%% \begin{equation}
%%   \left\{
%%   \begin{array}{l}
%%     \forall z \in C^-, \: \| z - \omega \|^2 \leq \rho^2 \\
%%     \forall z \in C^+, \: \| z - \omega \|^2 \geq \rho^2 \\
%%   \end{array}
%%   \right.
%% \end{equation}

%% \begin{qu}
%% Implement a function that checks whether $C$ is a digital plane or not. 
%% \end{qu}


\begin{qu}
  Perform a running time analysis of your recognition function.  
  TODO
  %% \begin{itemize}
  %% \item Implement a function that constructs the contour of a disk of a given radius.  
  %% \item Output the running time of your recognition function for disks of increasing radius. 
  %% \item Plot the graph of the running times against the size of the contour.
  %% \item Do you observe the expected linear-time complexity ?
  %% \end{itemize}	 
\end{qu}	

\section{Extra works}

\begin{qu}
Modify your recognition procedure in order to have an on-line algorithm, 
which takes input points one by one 
 and updates the current separating plane on the fly.
What is the time complexity of your procedure ?
\end{qu}

\begin{qu}
Use your on-line procedure to partition a digital surface into pieces of digital planes. 
\end{qu}



\end{document}
